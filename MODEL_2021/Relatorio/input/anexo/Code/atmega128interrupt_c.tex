\begin{lstlisting}[language=C, caption={atmega128interrupt.c}, label=atmega128interrupt-c, captionpos=b]
/*************************************************************************
ATMEGA128INTERRUPT
Author: Sergio Santos 
<sergio.salazar.santos@gmail.com>
Hardware: ATmega128
Date: 25102020
Comment:
Stable
*************************************************************************/
/***Preamble Inic***/
#ifndef F_CPU
	#define F_CPU 16000000UL
#endif
/***Library***/
#include <avr/io.h>
#include <avr/interrupt.h>
#include <avr/pgmspace.h>
#include <stdarg.h>
#include "atmega128interrupt.h"
/***Constant & Macro***/
#ifndef GLOBAL_INTERRUPT_ENABLE
	#define GLOBAL_INTERRUPT_ENABLE 7
#endif
#if defined(__AVR_ATmega64__) || defined(__AVR_ATmega128__)	
	/***/
	#define ATMEGA_INTERRUPT
	#define External_Interrupt_Control_Register_A EICRA
	#define External_Interrupt_Control_Register_B EICRB
	#define External_Interrupt_Mask_Register EIMSK
	#define External_Interrupt_Flag_Register EIFR
	#define MCU_Control_Status_Register MCUCSR
	#define MCU_Control_Status_Register_Mask 0X1F
#else
	#error "Not Atmega 128"
#endif
/***Gloabal File Variables***/
/***Header***/
void INTERRUPT_set(uint8_t channel, uint8_t sense);
void INTERRUPT_off(uint8_t channel);
void INTERRUPT_on(uint8_t channel);
uint8_t INTERRUPT_reset_status(void);
/***Procedure & Function***/
INTERRUPT INTERRUPTenable(void)
/***
Setup blank
***/
{
INTERRUPT interrupt;
External_Interrupt_Mask_Register = 0X00;
/******/
interrupt.set=INTERRUPT_set;
interrupt.off=INTERRUPT_off;
interrupt.on=INTERRUPT_on;
interrupt.reset_status=INTERRUPT_reset_status;
return interrupt;
}
uint8_t INTERRUPT_reset_status(void)
{
uint8_t reset, ret=0;
reset=(MCU_Control_Status_Register & MCU_Control_Status_Register_Mask);
switch(reset){
	case 1: // Power-On Reset Flag
		ret=0;
		MCU_Control_Status_Register &= ~(1<<PORF);
		break;
	case 2: // External Reset Flag
		MCU_Control_Status_Register &= ~(1<<EXTRF);
		ret=1;
		break;
	case 4: // Brown-out Reset Flag
		MCU_Control_Status_Register &= ~(1<<BORF);
		ret=2;
		break;
	case 8: // Watchdog Reset Flag
		MCU_Control_Status_Register &= ~(1<<WDRF);
		ret=3;
		break;
	case 16: // JTAG Reset Flag
		MCU_Control_Status_Register &= ~(1<<JTRF);
		ret=4;
		break;
	default: // clear all status
		MCU_Control_Status_Register &= ~(MCU_Control_Status_Register_Mask);
		break;
}
return ret;
}
void INTERRUPT_set(uint8_t channel, uint8_t sense)
{
switch( channel ){
	case 0: 
		External_Interrupt_Mask_Register &= ~(1<<INT0);
		External_Interrupt_Control_Register_A &= ~((1<<ISC01) |
		(1<<ISC00));
		switch(sense){
			case 0: // The low level of INTn generates an interrupt request.
			case 1: // The low level of INTn generates an interrupt request.
				break;
			case 2: 
			// The falling edge of INTn generates asynchronously an interrupt request.
				External_Interrupt_Control_Register_A |= (1<<ISC01);
				break;
			case 3: 
			// The rising edge of INTn generates asynchronously an interrupt request.
				External_Interrupt_Control_Register_A |= ((1<<ISC01) | (1<<ISC00));
				break;
			default: // The low level of INTn generates an interrupt request.
				break;
		}
	External_Interrupt_Mask_Register |= (1<<INT0);
	break;
	case 1:
		External_Interrupt_Mask_Register &= ~(1<<INT1);
		External_Interrupt_Control_Register_A &= ~((1<<ISC11) | (1<<ISC10));
		switch(sense){
			case 0: // The low level of INTn generates an interrupt request.
			case 1: // The low level of INTn generates an interrupt request.
				break;
			case 2: 
			// The falling edge of INTn generates asynchronously an interrupt request.
				External_Interrupt_Control_Register_A |= (1<<ISC11);
				break;
			case 3: 
			// The rising edge of INTn generates asynchronously an interrupt request.
				External_Interrupt_Control_Register_A |= ((1<<ISC11) | (1<<ISC10));
				break;
			default: // The low level of INTn generates an interrupt request.
				break;
		}
		External_Interrupt_Mask_Register |= (1<<INT1);
		break;
	case 2:
		External_Interrupt_Mask_Register &= ~(1<<INT2);
		External_Interrupt_Control_Register_A &= ~((1<<ISC21) | (1<<ISC20));
		switch(sense){
			case 0: // The low level of INTn generates an interrupt request.
			case 1: // The low level of INTn generates an interrupt request.
				break;
			case 2: 
			// The falling edge of INTn generates asynchronously an interrupt request.
				External_Interrupt_Control_Register_A |= (1<<ISC21);
				break;
			case 3: 
			// The rising edge of INTn generates asynchronously an interrupt request.
				External_Interrupt_Control_Register_A |= ((1<<ISC21) | (1<<ISC20));
				break;
			default: // The low level of INTn generates an interrupt request.
				break;
		}
		External_Interrupt_Mask_Register |= (1<<INT2);
		break;
	case 3:
		External_Interrupt_Mask_Register &= ~(1<<INT3);
		External_Interrupt_Control_Register_A &= ~((1<<ISC31) | (1<<ISC30));
		switch(sense){
			case 0: // The low level of INTn generates an interrupt request.
			case 1: // The low level of INTn generates an interrupt request.
				break;
			case 2: 
			// The falling edge of INTn generates asynchronously an interrupt request.
				External_Interrupt_Control_Register_A |= (1<<ISC31);
				break;
			case 3: 
			// The rising edge of INTn generates asynchronously an interrupt request.
				External_Interrupt_Control_Register_A |= ((1<<ISC31) | (1<<ISC30));
				break;
			default: // The low level of INTn generates an interrupt request.
				break;
		}
		External_Interrupt_Mask_Register |= (1<<INT3);
		break;
	case 4:
		External_Interrupt_Mask_Register &= ~(1<<INT4);
		External_Interrupt_Control_Register_B &= ~((1<<ISC41) | (1<<ISC40));
		switch(sense){
			case 0: // The low level of INTn generates an interrupt request.
				break;
			case 1: // Any logical change on INTn generates an interrupt request
				External_Interrupt_Control_Register_B |= (1<<ISC40);
				break;
			case 2: 
			// The falling edge between two samples of INTn generates an interrupt request.
				External_Interrupt_Control_Register_B |= (1<<ISC41);
				break;
			case 3: 
			// The rising edge between two samples of INTn generates an interrupt request.
				External_Interrupt_Control_Register_B |= ((1<<ISC41) | (1<<ISC40));
				break;
			default: // The low level of INTn generates an interrupt request.
				break;
		}
		External_Interrupt_Mask_Register |= (1<<INT4);
		break;
	case 5:
		External_Interrupt_Mask_Register &= ~(1<<INT5);
		External_Interrupt_Control_Register_B &= ~((1<<ISC51) | (1<<ISC50));
		switch(sense){
			case 0: // The low level of INTn generates an interrupt request.
				break;
			case 1: // Any logical change on INTn generates an interrupt request
				External_Interrupt_Control_Register_B |= (1<<ISC50);
				break;
			case 2: 
			// The falling edge between two samples of INTn generates an interrupt request.
				External_Interrupt_Control_Register_B |= (1<<ISC51);
				break;
			case 3: 
			// The rising edge between two samples of INTn generates an interrupt request.
				External_Interrupt_Control_Register_B |= ((1<<ISC51) | (1<<ISC50));
				break;
			default: // The low level of INTn generates an interrupt request.
				break;
		}
		External_Interrupt_Mask_Register |= (1<<INT5);
		break;
	case 6:
		External_Interrupt_Mask_Register &= ~(1<<INT6);
		External_Interrupt_Control_Register_B &= ~((1<<ISC61) | (1<<ISC60));
		switch(sense){
			case 0: // The low level of INTn generates an interrupt request.
				break;
			case 1: // Any logical change on INTn generates an interrupt request
				External_Interrupt_Control_Register_B |= (1<<ISC60);
				break;
			case 2: 
			// The falling edge between two samples of INTn generates an interrupt request.
				External_Interrupt_Control_Register_B |= (1<<ISC61);
				break;
			case 3: 
			// The rising edge between two samples of INTn generates an interrupt request.
				External_Interrupt_Control_Register_B |= ((1<<ISC61) | (1<<ISC60));
				break;
			default: // The low level of INTn generates an interrupt request.
				break;
		}
		External_Interrupt_Mask_Register |= (1<<INT6);
		break;
	case 7:
		External_Interrupt_Mask_Register &= ~(1<<INT7);
		External_Interrupt_Control_Register_B &= ~((1<<ISC71) | (1<<ISC70));
		switch(sense){
			case 0: // The low level of INTn generates an interrupt request.
				break;
			case 1: // Any logical change on INTn generates an interrupt request
				External_Interrupt_Control_Register_B |= (1<<ISC70);
				break;
			case 2: 
			// The falling edge between two samples of INTn generates an interrupt request.
				External_Interrupt_Control_Register_B |= (1<<ISC71);
				break;
			case 3: 
			// The rising edge between two samples of INTn generates an interrupt request.
				External_Interrupt_Control_Register_B |= ((1<<ISC71) | (1<<ISC70));
				break;
			default: // The low level of INTn generates an interrupt request.
				break;
		}
		External_Interrupt_Mask_Register |= (1<<INT7);
		break;
	default:
		External_Interrupt_Mask_Register = 0X00;
		break;
}
}
void INTERRUPT_off(uint8_t channel)
{
	switch( channel ){
		case 0: // disable
		External_Interrupt_Mask_Register &= ~(1<<INT0);
		break;
		case 1: // disable
		External_Interrupt_Mask_Register &= ~(1<<INT1);
		break;
		case 2: // disable
		External_Interrupt_Mask_Register &= ~(1<<INT2);
		break;
		case 3: // disable
		External_Interrupt_Mask_Register &= ~(1<<INT3);
		break;
		case 4: // disable
		External_Interrupt_Mask_Register &= ~(1<<INT4);
		break;
		case 5: // disable
		External_Interrupt_Mask_Register &= ~(1<<INT5);
		break;
		case 6: // disable
		External_Interrupt_Mask_Register &= ~(1<<INT6);
		break;
		case 7: // disable
		External_Interrupt_Mask_Register &= ~(1<<INT7);
		break;
		default: // all disable
		External_Interrupt_Mask_Register = 0X00;
		break;
	}
}
void INTERRUPT_on(uint8_t channel)
{
	switch( channel ){
		case 0:
		External_Interrupt_Mask_Register |= (1<<INT0);
		break;
		case 1:
		External_Interrupt_Mask_Register |= (1<<INT1);
		break;
		case 2:
		External_Interrupt_Mask_Register |= (1<<INT2);
		break;
		case 3:
		External_Interrupt_Mask_Register |= (1<<INT3);
		break;
		case 4:
		External_Interrupt_Mask_Register |= (1<<INT4);
		break;
		case 5:
		External_Interrupt_Mask_Register |= (1<<INT5);
		break;
		case 6:
		External_Interrupt_Mask_Register |= (1<<INT6);
		break;
		case 7:
		External_Interrupt_Mask_Register |= (1<<INT7);
		break;
		default: // all disable
		External_Interrupt_Mask_Register = 0X00;
		break;
	}
}
/***Interrupt***/
// cross out the ones being used and redefine in main
ISR(INT0_vect){ }
//ISR(INT1_vect){ }
ISR(INT2_vect){ }
ISR(INT3_vect){ }
ISR(INT4_vect){ }
ISR(INT5_vect){ }
ISR(INT6_vect){ }
ISR(INT7_vect){ }
/***EOF***/
\end{lstlisting}
%\end{verbatimtab}
%%%%%%%%%%%%%%%%%%%%%%%%%%%%%%%%%%%%%%%%%%%%%%%%%%%%%%%%%%%%%%%%
