\section{Conclusões}
\qquad Pelo resultado do inquérito feito pode-se dizer que a S.Roque aponta para uma cultura predominantemente \textcolor{blue}{inovadora}, com um grau elevado de incerteza devido ter muito poucas amostras. Com este inquérito, deu para perceber que cada individuo tem uma perceção diferente da organização, o que leva a crer que a S.Roque não tem uma cultura forte. O estilo de liderança é mais orientado ás tarefas, ou seja, uma liderança transacional ou instrumental.
\emptyline
%%%%%%%%%%%%%%%%%%%%%%%%%%%
A S.Roque tem orgulho em afirmar que é o primeiro no nosso mercado interno e destaca-se internacionalmente. O seu sucesso poderá ser medido recorrendo ao \textcolor{blue}{benchmarking} que o confirma. Também a nível de satisfação dos seus clientes não obtive nenhuma informação pelo método \textcolor{blue}{scorecard management}, e no seu site não tem indicações relacionadas. O \textcolor{blue}{Banco de Portugal} também é outra fonte onde é possível verificar o sucesso da organização quanto ao seu histórico de capital. O desempenho da organização pode no final ser demonstrado através da sua eficácia e eficiência.
\emptyline
As melhores práticas podem ser consultadas em fontes como a American National Standards Institute (ANSI) ou a Canadian Standards Association (CSA), ou internacional, tais como as normas ISO ou Institute of Electrical and Electronics Engineers (IEEE), e IEC.\\
%%%%%%%%%%%%%%%%%%%%%%%%%%
Na S.Roque como se trabalha com equipamentos elétricos e eletrónicos, tem que respeitar normas e diretivas em vigor de cada produto, tendo sempre em consideração a qualidade. Tem que ter indivíduos com formação especifica, com autonomia no seu trabalho. Os produtos só são considerados prontos para o mercado depois de montados e testados nas suas instalações, assim têm uma boa base de garantia, evitando erros e reclamações.\\
Os colaboradores tem um ambiente sem pressões e com tarefas pré-determinadas, o único ponto desfavorável talvez é de ser trabalho monótono, tendo que se criar tarefas rotativas, que nem sempre é possível por ter diferentes especialidades.
\emptyline
%%%%%%%%%%%%%%%%%%%%%%%%%%%
O estudo Ogbonna \& Harris também nos demonstrou que um líder participativo, isto é, o que permite que os seus subordinados possam influenciar nas suas decisões pelas suas opiniões e contribuições e o líder de suporte que está focado ao bem estar dos colaboradores, tendo uma atitude amigável e simpática são mais prósperos a obter resultados positivos.\\
O líder instrumental tendo um desempenho negativo, pois esta orientado em criar expectativas, estabelecer procedimentos e atribuir tarefas.\\
A S.Roque aqui pode tirar partido destas observações, não retirando a importância do papel da comunicação, a motivação e resolução de problemas.
A satisfação dos colaboradores e perspectivas de desenvolverem, são fatores que contribuem positivamente. Como se sabe não vale a pena remar o barco se não se sabe o destino.
\emptyline
%%%%%%%%%%%%%%%%%%%%%%%%%%%
Neste relatório a matéria abordada é a \textcolor{blue}{Cultura Organizacional}, com varias indicações comprovadas e medidas, que deve servir de apoio às organizações, e foi demonstrado na sua elaboração. Esta disciplina é muito enriquecedora, existindo uma vasta informação literária com suporte bastante alargado na \textit{internet}, para os que quiserem desenvolver os temas abordados.
%%%%%%%%%%%%%%%%%%%%%%%%%%%%%%%%%%%%%%%%%%%%%%%%%%%%%%%%%%%%%%%%%%%%%%%%%%%%%%%%%
