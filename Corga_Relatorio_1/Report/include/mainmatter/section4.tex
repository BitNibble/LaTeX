\section{Cultura Organizacional}
\qquad Os fundadores da organização têm um grande impacto no processo de formação da cultura organizacional através da imposição das suas crenças e suposições no grupo. A adoção das crenças, valores e suposições que formam a cultura da organização é depois reforçada pelos vários comportamentos primários dos lideres.\emptyline
%%%%%%%%%%%%%%%%%%%%%%%%%%
Planeamento Estratégico:
%%%%%%%%%%%%%%%%%%%%%%%%%%
\subsection{Missão}
A S.Roque tem como missão a constante inovação e criação de produtos de excelência na área da estamparia têxtil à peça. Para tal, aposta em múltiplos vetores complementares: tecnologia, qualidade e recursos humanos especializados. Estimula de forma persistente a sua veia empreendedora e internacional, promovendo para isso o contínuo aperfeiçoamento do seu serviço, em qualquer parte do mundo, mantendo-se fiel aos princípios éticos e de sustentabilidade.
%%%%%%%%%%%%%%%%%%%%%%%%%
\subsection{Visão}
Trilhar um percurso sustentável de inovação, de expansão internacional, de excelência em todas as soluções que lançamos para o mercado, de qualidade absoluta, para nos mantermos como líder na nossa área de negócio.
%%%%%%%%%%%%%%%%%%%%%%%%%%
\subsection{Valores}
\begin{itemize}
\setlength\itemsep{-0.3em}
\item Ação dentro dos princípios morais e éticos da empresa para com os seus \textit{stakeholders}.
\item Atuação sempre no interesse dos nossos parceiros de forma a promover a sua satisfação e fidelização.
\item Excelência conseguida através de trabalho de equipa, competência e responsabilidade.
\item Qualidade absoluta.
\item Inovação promovida pelo ADN empreendedor da S. Roque.
\item Sustentabilidade ambiental e segurança.
\end{itemize}\par
%%%%%%%%%%%%%%%%%%%%%%%%%%
Além da sua definição estratégica da cultura organizacional pode se fazer objetivos pelo método \textcolor{blue}{\textbf{SMART}}, para ajudar a definir a sua missão.
%%%%%%%%%%%%%%%%%%%%%%%%%%%%%%%%%%%%%%%%%%%%%%%%%%%%%%%%%%%%%%%%%%%%%%%%%%%%%%%%%
\subsection{Analise SWOT:}
A analise \textbf{SWOT} consiste em ver a organização em dois aspetos do ambiente onde estão inseridos, o interno e externo, e duas vertentes na qual terá influencia positiva ou negativa. O ambiente externo também tem duas vertentes que consiste no ambiente transacional e contextual com influencia \textcolor{blue}{PEST}, isto é, política, económica, social e tecnológico. Uma ferramenta muito útil para auxiliar na planificação.
\begin{itemize}
\setlength\itemsep{-0.5em}
\item \textcolor{purple}{I}nterno
\begin{itemize}
\setlength\itemsep{-0.3em}
\item \textcolor{orange}{S}trength (forças)\\
- Tem liderança no mercado interno, e reconhecimento internacional\\
- As maquinas são proprietárias\\
- Tem um mercado internacional\\
- Mão de obra qualificada é barata\\
- Tem um controlo de qualidade\\
- Dá suporte ao cliente\\
- Oportunidade de entrar noutros mercados
\item \textcolor{orange}{W}eakness (fraquezas)\\
- Mercado saturado internamente\\
- Localidade de produção isolado\\
- Dificuldade em obter mão de obra\\
- Portugal é reconhecido por ter má gestão\\
- Dificuldade em obter mão de obra qualificada
\end{itemize}
\item \textcolor{purple}{E}xterno
\begin{itemize}
\setlength\itemsep{-0.3em}
\item \textcolor{orange}{O}pportunity (oportunidades)\\
- Fácil acesso ao credito\\
- Inserido num país europeu com mão de obra barata\\
- Inserido numa sociedade femininista\\
- Ajudas do estado para o desenvolvimento (ex: programa 2020)\\
- Tecnologia mais recentes ao dispor
\item \textcolor{orange}{T}hreats (ameaças)\\
- A competir com mercado internacional mais forte\\
- Areá têxtil sob ameaça\\
- Sociedade que evita a incerteza
\end{itemize}
\end{itemize}
%%%%%%%%%%%%%%%%%%%%%%%%%%%%%%%%%%%%%%%%%%%%%%%%%%%%%%%%%%%%%%%%%%%%%%%%%%%%%%%%%
